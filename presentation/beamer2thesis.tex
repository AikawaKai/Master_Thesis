\documentclass{beamer}
\usetheme[titlepagelogo=minerva2,% Logo for the first page
						language=italian
                        ]{TorinoTh}
                        
\usepackage[beamer,customcolors]{hf-tikz}
\hfsetfillcolor{alerted text.fg!10}
\hfsetbordercolor{alerted text.fg}

\author{Marco Odore}
\rel{Prof. Giorgio Valentini}
\assistantsupervisor{Dr. Marco Notaro}
\title[Metodi di Ensemble Gerarchici]{Metodi di Ensemble Gerarchici per la Predizione Strutturata della Funzione delle Proteine}
\ateneo{Università Degli Studi Di Milano}
\date{10 Luglio 2018}

\begin{document}
\titlepageframe
\begin{tframe}{Il Problema della Predizione Delle Proteine}
  % 
  \begin{columns}
    %
    \begin{column}{.70\textwidth}
      \minipage[c][0.30\textheight][s]{\columnwidth}
	   \begin{itemize}	
      \item Le proteine sono polimeri (macromolecole biologiche) composte da amminoacidi, che sono codificate dai geni (porzioni di genoma) presenti nel DNA di ogni essere vivente. 

      \onslide<2->

      \item Ogni gene può codificare più proteine.

      \onslide<3->

      \item Si stima che l'essere umano possegga tra i 20.000 - 25.000 geni.
		
      \end{itemize}
      \vfill
      \onslide<4->
      \begin{tabular}{|p{0.9\textwidth}}
        Use \texttt{minipage} for top aligned images, and
        \texttt{parbox} for vertically centered images.
      \end{tabular}


      \endminipage      
    \end{column}
    %
    \begin{column}{.30\textwidth}

      % for top aligned images use minipage
      \only<1-3>{
        \minipage[c][0.8\textheight][s]{\columnwidth}

        \onslide<1->    

        \only<1-3>{
          \begin{figure}
            \centering
            \includegraphics<1>[scale=0.3]{%
              img/DNA.png} %
            \includegraphics<2-3>[scale=0.3]{%
              img/DNA.png} %
        \end{figure}}

        \only<3>{
          \begin{figure}
            \centering
            \includegraphics[scale=0.3]{%
              img/DNA.png} %
        \end{figure}}

        \endminipage
      }   

      % for vertically centered images use parbox
      \only<4>{
        \parbox[c][0.8\textheight][c]{\columnwidth}{
          \begin{figure}
            \centering
            \includegraphics[scale=0.3]{%
              img/DNA.png} %
          \end{figure}
        }
      }

    \end{column}
  \end{columns}

\end{tframe}


\begin{frame}[t,fragile]{Configuration}
\begin{itemize}
\item The configuration of the standard theme is:
\begin{itemize}
\item \verb!language=italian!
\item \verb!coding=utf8x!
\item \verb!titlepagelogo=name-of-the-logo!
\item \verb!bullet=circle!
\item \verb!pageofpages=of!
\item \verb!titleline=true!
\item \verb!color=blue!
\item \verb!secondcandidate=false!
\item \verb!secondlogo=false!
\end{itemize}
\item Most of them, actually everyone except the \highlight{titlepagelogo}, can be omitted if there are no modifications
\end{itemize}
\end{frame}

\begin{frame}[fragile]{Behavior of alerts}
Each color theme requires different colors to highlight words. To insert alerts by using the \emph{itemize} environment, you can exploit:
\begin{verbatim}
\begin{itemize}
\item<+-| alert@+> Apple
\item<+-| alert@+> Peach
\end{itemize}
\end{verbatim}
For example:
\begin{itemize}
\item<+-| alert@+> Apple
\item<+-| alert@+> Peach
\end{itemize}
\end{frame}

\begin{frame}[fragile]{Another way to highlight words}
If you want to highlight your text out of the enviroment \emph{itemize}, Beamer2Thesis offers you the following possibilities:
\begin{itemize}
\item the standard command \verb!\alert{text}!: it simply highlights your \alert{text}
\item the command \verb!\highlight{text}!: it highlights your \highlight{text} setting it in italic
\item the command \verb!\highlightbf{text}!: it highlights your \highlightbf{text} setting it in bold
\end{itemize}
Of course, the color used, is set accordingly to your choice in the configuration phase.
\end{frame}

\begin{frame}[fragile]{Highlighting formulas}
\begin{itemize}
\item The package \href{http://www.ctan.org/pkg/hf-tikz}{hf-tikz} allows to highlight formulas and formula parts in Beamer with overlay specifications 
\item The adaptation of colors to the theme could be done in this way:
\begin{verbatim}
\usepackage[beamer,customcolors]{hf-tikz}
\hfsetfillcolor{alerted text.fg!10}
\hfsetbordercolor{alerted text.fg}
\end{verbatim}
\item \highlight{Two compilation runs} are required to get the right result!
\item Read the package documentation to find more options; an example will be provided in the next frame.
\end{itemize}
\end{frame}

\begin{frame}[fragile]{Highlighting formulas (II)}
\begin{itemize}
\item Example:
\[\tikzmarkin<2->{a}x+\tikzmarkin<1>{b}y\tikzmarkend{b}=10\tikzmarkend{a}\]
\item<2-> Code:
\begin{verbatim}
\[\tikzmarkin<2->{a}x+
  \tikzmarkin<1>{b}y\tikzmarkend{b}
  =10\tikzmarkend{a}\]
\end{verbatim}
\end{itemize}
\end{frame}

\input{content_end}
\end{document}
