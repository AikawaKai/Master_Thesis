\documentclass{beamer}
\usetheme[titlepagelogo=minerva2,% Logo for the first page
						language=italian
                        ]{TorinoTh}
                        
\usepackage[beamer,customcolors]{hf-tikz}
\hfsetfillcolor{alerted text.fg!10}
\hfsetbordercolor{alerted text.fg}

\author{Marco Odore}
\rel{Prof. Giorgio Valentini}
\assistantsupervisor{Dr. Marco Notaro}
\title[Metodi di Ensemble Gerarchici]{Metodi di Ensemble Gerarchici per la Predizione Strutturata della Funzione delle Proteine}
\ateneo{Università Degli Studi Di Milano}
\date{10 Luglio 2018}

\begin{document}
\titlepageframe
\begin{tframe}{Central Dogma}
  % 
  \begin{columns}
    %
    \begin{column}{.65\textwidth}
      \minipage[c][0.4\textheight][s]{\columnwidth}
	   \begin{itemize}	
	  \item All'interno delle molecole di DNA di ogni essere vivente esistono diverse migliaia di geni.   
      \onslide<2->
	  \item Si stima che per l'essere umano il DNA possegga tra i 20.000 - 25.000 geni.
      \onslide<3->
      \item Ogni gene all'interno del DNA è capace di codificare più proteine.	
      \onslide<4->
      \item Ogni proteina è responsabile di una o più funzioni all'interno delle cellule degli esseri viventi.
      \end{itemize}
      \endminipage      
    \end{column}
    %
    \begin{column}{.35\textwidth}

      % for top aligned images use minipage
      \only<1-4>{
        \minipage[c][0.4\textheight][s]{\columnwidth}

        \onslide<1->    

        \only<1-4>{
          \begin{figure}
            \centering
            \includegraphics<1>[scale=0.2]{img/DNA.png} %         
            \includegraphics<2>[scale=0.3]{img/humandna.png} %
            \includegraphics<3>[scale=0.2]{img/centraldogma.png}%
            \includegraphics<4>[scale=0.5]{img/funzioni.jpg}
        \end{figure}}


        \endminipage
      }   

      % for vertically centered images use parbox

    \end{column}
  \end{columns}

\end{tframe}

\begin{tframe}{La funzione delle proteine}


      Le proteine sono molecole biologiche composte da amminoacidi, e le funzioni che svolgono sono molteplici:
	   \begin{itemize}	
	   \onslide<2->
	   \item Metaboliche, ad esempio per la produzione di energia
	   \onslide<3->
		\item Di trascrizione, sintesi e processamento delle proteine stesse
		\onslide<4->		
		\item Di trasporto
		\onslide<5->
		\item Di comunicazione intra o intercellulare
		\onslide<6->
		\item Di ciclo della cellula, ad esempio per la divisione e riproduzione cellulare
      \end{itemize}
      \onslide<7->
      Per molte specie le funzioni di moltissimi geni (e quindi delle corrispettive proteine codificate) è \highlightbf{sconosciuta o parzialmente nota}.

\end{tframe}

\begin{tframe}{\small Il problema della predizione della funzione delle proteine 1/3}
  % 
  \begin{columns}
    %
    \begin{column}{.75\textwidth}
      \minipage[c][0.4\textheight][s]{\columnwidth}
	   \begin{itemize}	
	   \onslide<1->
	  \item L'individuazione della funzione delle proteine attraverso le analisi con sperimentazione diretta in laboratorio è \highlightbf{costosa} e richiede \highlightbf{molto tempo}
	  \onslide<2->
	  \item Esistono centinaia di funzioni a cui poter associare un gene/proteina \highlightbf{(problema multiclasse)}
	  \onslide<3->
	  \item Ad ogni gene/proteina possono essere associate diverse funzioni contemporaneamente \highlightbf{(problema multietichetta)}
      \onslide<4->
	  \item Il quantitativo di dati genomici cresce molto rapidamente.
      \end{itemize}
      \onslide<5->
      La \highlightbf{classificazione manuale} delle proteine è quindi infattibile.
      \endminipage      
    \end{column}
    %
    \begin{column}{.25\textwidth}

      % for top aligned images use minipage
      \only<1-5>{
        \minipage[c][0.4\textheight][s]{\columnwidth}

        \onslide<1->    

        \only<1-5>{
          \begin{figure}
            \centering
            \includegraphics<1>[scale=0.15]{img/lab3.jpg} %         
            \includegraphics<2>[scale=0.2]{img/multiclass.png} %
            \includegraphics<3>[scale=0.3]{img/multilabel.png}%
            \includegraphics<4>[scale=0.16]{img/growth.jpg}
        \end{figure}}


        \endminipage
      }   

      % for vertically centered images use parbox

    \end{column}
  \end{columns}

\end{tframe}

\begin{tframe}{\small Il problema della predizione della funzione delle proteine 2/3}
\begin{itemize}
\onslide<1->
\item A complicare ulteriormente il problema è il modo in cui sono \emph{relazionate} tra loro le funzioni delle proteine.
\onslide<2->
\item Esistono infatti due tassonomie principali per l'organizzazione delle classi:
\begin{itemize}

\onslide<3->
\item \highlightbf{Gene Ontology} (GO):  che organizza le funzioni come un grafo diretto aciclico (DAG), varia per ogni specie, e possiede tre ontologie differenti.
\onslide<4->
\item \highlightbf{Functional Catalogue} (FunCat): che è organizzato invece come un albero, non varia in base alle specie, e descrive le funzioni in maniera più sintetica rispetto alla Gene Ontology.
\end{itemize}
\end{itemize}  

\end{tframe}

\begin{tframe}{\small Il problema della predizione della funzione delle proteine (GO) 3/3}
\begin{itemize}
\onslide<1->
\item Data la granularità e specificità superiori della GO e il suo largo utilizzo nella comunità scientifica, all’interno della tesi ci si è soffermati sulla predizione delle sue funzioni.
\onslide<2->
\item Tale tassonomia presenta tre ontologie (e quindi tre DAG) principali:
\begin{itemize}
\onslide<3->
\item \highlightbf{Processo Biologico} (BP): descrive i processi ad alto livello, come insieme di diverse attività molecolari.
\onslide<4->
\item \highlightbf{Funzione Molecolare} (MF): descrive le funzioni di specifici prodotti genici.
\onslide<5->
\item \highlightbf{Componente Cellulare} (CC):il luogo all’interno della cellula nelle quali avviene la funzione genica.
\end{itemize}
\end{itemize}  

\end{tframe}

\begin{tframe}{La predizione automatica}
Per gestire il problema della predizione della funzione delle proteine si rende quindi necessario un approccio \highlightbf{automatico}.

\end{tframe}


\begin{frame}[t,fragile]{Configuration}
\begin{itemize}
\item The configuration of the standard theme is:
\begin{itemize}
\item \verb!language=italian!
\item \verb!coding=utf8x!
\item \verb!titlepagelogo=name-of-the-logo!
\item \verb!bullet=circle!
\item \verb!pageofpages=of!
\item \verb!titleline=true!
\item \verb!color=blue!
\item \verb!secondcandidate=false!
\item \verb!secondlogo=false!
\end{itemize}
\item Most of them, actually everyone except the \highlight{titlepagelogo}, can be omitted if there are no modifications
\end{itemize}
\end{frame}

\begin{frame}[fragile]{Behavior of alerts}
Each color theme requires different colors to highlight words. To insert alerts by using the \emph{itemize} environment, you can exploit:
\begin{verbatim}
\begin{itemize}
\item<+-| alert@+> Apple
\item<+-| alert@+> Peach
\end{itemize}
\end{verbatim}
For example:
\begin{itemize}
\item<+-| alert@+> Apple
\item<+-| alert@+> Peach
\end{itemize}
\end{frame}

\begin{frame}[fragile]{Another way to highlight words}
If you want to highlight your text out of the enviroment \emph{itemize}, Beamer2Thesis offers you the following possibilities:
\begin{itemize}
\item the standard command \verb!\alert{text}!: it simply highlights your \alert{text}
\item the command \verb!\highlight{text}!: it highlights your \highlight{text} setting it in italic
\item the command \verb!\highlightbf{text}!: it highlights your \highlightbf{text} setting it in bold
\end{itemize}
Of course, the color used, is set accordingly to your choice in the configuration phase.
\end{frame}

\begin{frame}[fragile]{Highlighting formulas}
\begin{itemize}
\item The package \href{http://www.ctan.org/pkg/hf-tikz}{hf-tikz} allows to highlight formulas and formula parts in Beamer with overlay specifications 
\item The adaptation of colors to the theme could be done in this way:
\begin{verbatim}
\usepackage[beamer,customcolors]{hf-tikz}
\hfsetfillcolor{alerted text.fg!10}
\hfsetbordercolor{alerted text.fg}
\end{verbatim}
\item \highlight{Two compilation runs} are required to get the right result!
\item Read the package documentation to find more options; an example will be provided in the next frame.
\end{itemize}
\end{frame}

\begin{frame}[fragile]{Highlighting formulas (II)}
\begin{itemize}
\item Example:
\[\tikzmarkin<2->{a}x+\tikzmarkin<1>{b}y\tikzmarkend{b}=10\tikzmarkend{a}\]
\item<2-> Code:
\begin{verbatim}
\[\tikzmarkin<2->{a}x+
  \tikzmarkin<1>{b}y\tikzmarkend{b}
  =10\tikzmarkend{a}\]
\end{verbatim}
\end{itemize}
\end{frame}

\input{content_end}
\end{document}
